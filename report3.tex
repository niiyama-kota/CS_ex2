\documentclass{jsarticle}
\usepackage{amsmath}
\usepackage{url}
\begin{document}
\begin{titlepage}
    \begin{center}
        \vspace*{12pt}
        {\LARGE 計算機科学実験及び演習2(論理素子)}
        \vspace{12pt}\\
        {第1回}
        \vspace{60pt}\\
        {締め切り : 12月10日}
        \vspace{12pt}\\
        {提出日 : 12月10日}
    \end{center}
    \begin{flushright}
        \vspace*{300pt}
        {15班}
        \vspace{12pt}\\
        {氏名 : 新山公太}
        \vspace{3pt}\\
        {学籍番号 : 1029300562}
        \vspace{3pt}\\
        {共同実験者 : 須村允亮}
        \vspace{3pt}\\
        {\phantom{共同実験者 : }小西岳志}
    \end{flushright}
    
\end{titlepage}

\section{実験1 : オシロスコープの使い方}
\subsection{キャリブレーション}
\subsubsection{目的}
この実験の目的はオシロスコープに波形が表示されることを確認することである。また、オシロスコープの基本的な使い方を確認するための機会であった。
\subsubsection{方法}
オシロスコープの入力端子1に校正用信号端子をつなぐ。接地端子もオシロスコープの接地用の端子につなぐ。その後オシロスコープの電源を入れてオートセットのボタンを押す。
\subsubsection{結果}
別紙のグラフ、\{実験1.1 キャリブレーション\}を参照。
\subsubsection{考察}
結果からわかることとして、出力される信号は一定の間隔で二つの値が交互に出力されている。メモリの区切りは結果を観察しやすいような区切りになっているが、これはオートセットボタンの機能だと思われる。

\subsection{X-Y表示(異なる信号発生器をそれぞれ入力としたとき)}
\subsubsection{目的}
この実験では入力1の値の時の入力2の値をX-Y平面にプロットして結果を観察することが目的である。
\subsubsection{方法}
オシロスコープの入力端子1には信号発生器1の出力を、オシロスコープの入力端子2には信号発生器2の出力をつなぐ。
\subsubsection{結果}
別紙のグラフ、\{実験1.2.1 異なる信号発生器のX-Y表示\}を参照。
\subsubsection{考察}
得られた図形は時間とともに形を変えていったがこれは、異なる信号発生器を用いたためオシロスコープへの入力に位相差が生まれたからであると考えられる。今回の実験では信号発生器の電圧と周波数は一致させたのである位相差$\theta$におけるグラフを表す式は
\begin{equation}
    y = \sin{(\arcsin{x}+\theta)}
\end{equation}
と表される。(ただし、xは入力信号の電圧[$mV$],yは出力信号の電圧[$mV$]. また$\arcsin{x}$は$-1 < x < 1$ に二つの解を持つ。)
電圧や周波数が異なるときは、入力信号1、入力信号2の電圧に関しての式を連立して解けばよく、

\begin{equation}
    \left\{
    \begin{aligned}
        x = V_1\sin{(\omega_1 t + \phi_1)}\\
        y = V_2\sin{(\omega_2 t + \phi_2)}
    \end{aligned}
    \right.
\end{equation}
で$t$を消去すればよい。結果は一般にリサージュ曲線となる。

\subsection{X-Y表示(同じ信号発生器を入力としたとき)}
\subsubsection{目的}
上と同一のため割愛。
\subsubsection{方法}
オシロスコープの入力端子1,2どちらにも同じ信号発生器の出力をつなぐ。
\subsubsection{結果}
時間による変化はなく、常に$y = x$の形をしていた。
別紙のグラフ、\{実験1.2.2 同じ信号発生器のX-Y表示\}を参照。
\subsubsection{考察}
同じ信号発生器を入力としているため入力の電圧や周波数は当然一緒で、上の実験とは違って位相も常に同じであるために直線のグラフが得られたと考えられる。

\section{実験2 微分回路、積分回路}
\subsection{微分回路}
\subsubsection{目的}
微分回路の動作を確認する。
\subsubsection{手段}
ブレッドボード上で別紙の回路図、\{実験2.1 回路図(微分回路)\}を組み$V_1$をオシロスコープで観察する。
用いた電気素子の特性は、抵抗値:$10k\Omega$,容量:$510pF$.
信号発生器の電圧は$5.0V$,周波数は$10kHz$.
\subsubsection{結果}
別紙のグラフ、\{実験2.1 微分回路\}を参照。
\subsubsection{考察}
グラフの形や時定数などは理論的なものと整合性が取れているが、実験が終わってから$t=0$の時の$V_1$の値がおかしいことに気が付いた。具体的には、信号発生器の電圧を$5.0V$に設定しているので$t=0$の時には$5V$が出力されるべきである。しかし実際の結果では$10V$に近い値が出ており、これは理論的にはありえない。なぜならば、もしもこの値が正しいと仮定するならばコンデンサは電流の流れている向きに電圧をかけていて、そうだとすればコンデンサにかかる電圧は無限大に発散するがそれはありえないだろう。何が起こっているのかわからなかったので色々調べた結果、オシロスコープには入力の電圧を何倍かに引き延ばす機能があるらしいことが分かった。したがって今回記録したグラフにおいては、オシロスコープの入力の電圧が2倍に増幅されたのではないかと結論した。
\url{http://digital.ni.com/public.nsf/allkb/ECABFA1F1432430D86257153000FE588}によると回路中のインピーダンスが大きいとこのような現象が起こるとあり、今回使っている抵抗は$10K\Omega$でコンデンサの容量は$510pF$で周波数は$10kHz$であるから回路のインピーダンスは$32770\Omega$である。上の記事では$1M\Omega$を超えるときとあるので条件は満たしていないが、他に原因は思いつかなかった。

\subsection{積分回路}
\subsubsection{目的}
積分回路の動作を確認する。
\subsubsection{手段}
微分回路と同様に\{実験2.2 回路図(積分回路)\}を組んで$V_1$をオシロスコープで観察する。
なお用いた電気素子の特性は微分回路と同じである。
\subsubsection{結果}
別紙のグラフ、\{実験2.2 積分回路\}を参照。
\subsubsection{考察}
微分回路の時とは違いこちらは、十分時間がたった後の電圧が$5V$に近い値となっており、グラフの形と時定数も含めて理論に沿った結果が得られた。なお時定数については微分回路と一致していた。

\subsection{問題1}
微分回路、積分回路ともに実験で得られた時定数は$6.3\mu s$であった。理論値に関しては\\
$\tau = RC = 10\cdot 10^3 \cdot 510 \cdot 10^{-12} = 5.1 \cdot 10^{-6} = 5.1 [\mu s]$\\
であり、そこまでの誤差はなくおおむね上手くいっていると思われる。
ある程度の誤差が生まれてしまった原因として、まず微分回路なら立ち下り37\%の測り方が目分量であるために大雑把であること、各電気素子の特性の表示値からの誤差などがあるためだと推測される。

\section{実験3}
\subsection{ダイオードの直流特性(順方向降下電圧)}
\subsubsection{目的}
ダイオードの特性をグラフ化し、順方向効果電圧$V_{f}$と飽和電流$I_{s}$を求める。
\subsubsection{手段}
別紙の回路図\{回路図(ダイオードの直流特性)\}の回路を組み、抵抗にかかる電圧$V_{1}$とダイオードにかかる電圧$V_{2}$を測る。
なお、図に示したように接地している点がダイオードと抵抗の間にあり、今回はダイオードにかかる電圧は負で表されることに注意した。(つまりグラフはx軸で$\pi$だけ回転させれば配布された資料と同様の図形となる。)
このグラフは縦軸が電圧となっているが実際には電流を測りたい。オシロスコープは電圧を測るほうが得意な機器であるので、まずは電圧を測ってからそれを$R$で割ってやればよいのでグラフの形は変わらない。
使った電気素子の特性は、抵抗:$10\Omega$,信号発生器の電圧は$8V$で周波数は$6kHz$である。
\subsubsection{結果}
別紙のグラフ、\{実験3.2 ダイオードの直流特性\}を参照。
降下電圧は上手く求めることができたが、飽和電流に関しては観測できなかったため以下に述べる様に条件を変えて実験した。
\subsubsection{考察}
y軸が反転しているがひっくり返せば、よくみるダイオードの特性をしっかり表しており観測は成功したといえると思う。特に降下電圧に関しては理論値の範囲内に収まっているので上手くいったといえるだろう。しかし飽和電流の測定に関しては上手くいかなかったで、電圧を高く、周波数を小さくして再び測定をしてみた。周波数を小さくすることで1周期が長くなり、y軸の変化が見やすくなる。
\subsection{ダイオードの直流特性(飽和電流)}
\subsubsection{目的}
上と同じであるので割愛。
\subsubsection{手段}
上と同じ条件で信号発生器の電圧を$10V$,周波数を$3.2kHz$に変更した。
\subsubsection{結果}
別紙のグラフ\{実験3.2 ダイオードの直流特性\}を参照。
飽和電流を目視で確認することができる。
\subsubsection{考察}
信号発生器の条件を変えることで飽和電流を求めることができた。電圧が変わるのを遅くしたいために周波数を下げるというのはいい発想だったが、電圧も調整しなければオシロスコープに表示される波形を表す線が太くなっていってしまう。今回はうまく調整できて良かった。

\section{実験4}
\subsection{74HCOOの特性}
\subsubsection{目的}
CMOS論理素子74HCOOの特性を確かめる。
\subsubsection{手段}
CMOSを起動させるために乾電池から電力を供給してやる。さらに信号発生器から電圧$6.0V$,周波数$20kHz$の正弦波を出力する。別紙の回路図\{実験4 回路図(CMOS論理素子の特性)\}の回路を組む。
\subsubsection{考察}
CMOS論理回路は外部から電力の供給がないと論理素子として機能しなかった。同じ信号発生器から二つの入力を受け取って結果を返す。値が変わるところでは線が太くなってしまってよくは見えなかったけど、他の部分はきれいに2つの分かれていたので論理回路の特性を確認することができたと思う。
\subsection{74HCOOの各入力に対する出力}
\subsubsection{目的}
74HCOOがどんな回路かを特定するため。 
\subsubsection{手段}
ICトレーナーの機能を使って調べた。
\subsubsection{結果}
\begin{table}[htb]
    \begin{tabular}{|l|c|r|r} \hline
        入力1 & 入力2 & 出力 \\ \hline \hline
        0 & 0 & 1 \\ \hline
        0 & 1 & 1 \\ \hline
        1 & 0 & 1 \\ \hline
        1 & 1 & 0 \\ \hline
    \end{tabular}
\end{table}
\subsubsection{考察}
この真理値表はNAND回路の真理値表と一致している。真理値表が一致しているのでこの実験で使ったCMOS素子、74HCOOはNAND回路である。
\subsection{問題3}
上で述べたように真理値表がNAND回路と一致しているので、74HCOOはNAND関数を表しているといえる。

\section{問題4}
\subsection{問題4-1}
CMOS論理素子はそれ自体ではインバータとして動作する。また、CMOS論理素子からNAND回路を作ることができ、NAND回路は組み合わせることで任意の論理回路を表すことができるので任意の電気回路はCMOSだけで作ることが理論的に可能である。CMOSにはファンアウトと呼ばれる回路素子としての制約がある。これは一つの論理素子の出力に接続する後段の論理素子の数のことである。CMOSのファンアウトはDTLやTTLに比べて一般的に大きく、また動作速度も非常に速いが、あまりにつなぎすぎると動作速度が遅くなってしまう。CMOS回路技術によって集積回路の低消費電力化が可能になっており、大規模集積回路の実現が加速された。更にはCMOSにはノイズに強いという特性もある。弱点としては信号電圧がゆっくり変化するときには出力が安定しないことがあげられる。

CMOSは金属と半導体の間に薄い酸化膜が挟まれた構造を持つP型トランジスタとN型トランジスタを組み合わせて作られる。さらにはサイズが非常に小さいという特徴もあり、現在、大規模集積回路といえばCMOSといわれるまでに使われている。
\subsection{問題4-2}
シュミットトリガ論理素子は入力増加時の閾値電圧と入力減少時の閾値電圧が異なる特性を持ち、これはヒステリシス特性と呼ばれる。つまり、この回路ではある閾値$V_{H}$よりも大きな入力があった時にはHighを出力して、別の閾値$V_{L}$よりも小さな入力があった場合にはLowが出力されるような回路である。

この回路の使い道としては、先に述べたCMOSの出力が安定しない場合にヒステリシス特性をもつシュミットトリガ回路を使うことでノイズを軽減することができる。これはCMOSの出力を安定させることに限った話ではなく、論理素子のノイズを取り除くのに役に立つ論理素子であるといえる。

\begin{thebibliography}{3}
    \bibitem http://digital.ni.com/public.nsf/allkb/ECABFA1F1432430D86257153000FE588
    \bibitem https://www.fujitsu.com/jp/group/labs/resources/tech/techguide/list/cmos/p02.html
    \bibitem http://www.miyazaki-gijutsu.com/series4/densi0523.html
\end{thebibliography}
\end{document}